% -*- TeX-master: "ualib-part1.tex" -*-
%%% Local Variables: 
%%% mode: latex
%%% TeX-engine: 'xetex
%%% End:
\documentclass[a4paper,UKenglish,cleveref,autoref,thm-restate,12pt]{lipics-v2021-wjd}
\usepackage{wjd}
% \usepackage{fontspec}
\usepackage{comment}
\usepackage{proof-dashed}
\usepackage{unixode}
\usepackage[wjdsimple,links]{agda}
\usepackage{hyperref}

\hypersetup{
%     bookmarks=true,         % show bookmarks bar?
%     unicode=true,          % non-Latin characters in Acrobat’s bookmarks
%     pdftoolbar=true,        % show Acrobat’s toolbar?
%     pdfmenubar=true,        % show Acrobat’s menu?
%     pdffitwindow=false,     % window fit to page when opened
% %    pdfstartview={FitW},    % fits the width of the page to the window
    pdftitle={Agda UALib, Part 1: Foundation},    % title
    pdfauthor={William DeMeo},     % author
    pdfsubject={Agda Universal Algebra Library},   % subject of the document
%     pdfcreator={XeLaTeX with hyperref},
%     pdfproducer={},  % producer of the document
%     pdfkeywords={Universal algebra} {Equational logic} {Martin-Löf Type Theory} {Birkhoff’s HSP Theorem} {Formalization of mathematics} {Agda} {Proof assistant},
%     pdfnewwindow=true,      % links in new window
%     colorlinks=true,       % false: boxed links; true: colored links
%     linkcolor=red,          % color of internal links
%     citecolor=green,        % color of links to bibliography
%     filecolor=magenta,      % color of file links
%     urlcolor=cyan           % color of external links
}
\newif\ifnonbold % comment this line to restore bold universe symbols

% \usepackage[x11names, rgb]{xcolor}
\usepackage{tikz}
\usetikzlibrary{snakes,arrows,shapes}
\usepackage{amsmath}
% \documentclass{amsart}
% \usepackage{amsmath}

\newcommand{\inr}{\operatorname{inr}}
\newcommand{\inl}{\operatorname{inl}}
\newcommand{\DLO}{\operatorname{DLO}}
\newcommand{\<}{\langle}
\renewcommand{\>}{\rangle}
\renewcommand\S{\operatorname{S}}
\newcommand\B{\operatorname{B}}
\renewcommand\L{\operatorname{L}}
\newcommand\cM{\mathcal{M}}
\newcommand\cN{\mathcal{N}}
\newcommand\CD{\mathcal{D}}
\newcommand\N{\mathbb{N}}
\newcommand\Z{\mathbb{Z}}
\newcommand\Th{\operatorname{Th}}
\newcommand\cl{\mathbf{cl}}
\newcommand\lt{\mathbf{lt}}
\newcommand\bfall{\boldsymbol{\forall}}
\newcommand\bfex{\boldsymbol{\exists}}

%%%%%%%%%%%%%% TITLE %%%%%%%%%%%%%%%%%%%%%%%%%%%%%%%%%%%%%%%%%%%%%%%%%%%%%%%%%%%%%%%%
\title{Model Theory Nomenclature}% and quotients}
%The Agda Universal Algebra Library, Part 1: Foundation. Equality, extensionality, truncation, and dependent types for relations and algebras


%%%%%%%%%%%%%% AUTHOR %%%%%%%%%%%%%%%%%%%%%%%%%%%%%%%%%%%%%%%%%%%%%%%%%%%%%%%%%%%%%%%%
\author{William DeMeo}
       {Department of Algebra, Charles University in Prague \and \url{https://williamdemeo.gitlab.io}}
       {williamdemeo@gmail.com}{https://orcid.org/0000-0003-1832-5690}{}
%\authorrunning{W.~J.~DeMeo}

\copyrightfooter

% \pagestyle{fancy}
% \renewcommand{\sectionmark}[1]{\markboth{#1}{}}
% \fancyhf{}
% \fancyhead[ER]{\sffamily\bfseries \leftmark}
% \fancyhead[OL]{\sffamily\bfseries Agda UALib, Part 1: Foundation}
% \fancyhead[EL,OR]{\sffamily\bfseries \thepage}

% \ccsdesc[500]{Theory of computation~Logic and verification}
% \ccsdesc[300]{Computing methodologies~Representation of mathematical objects}
% \ccsdesc[300]{Theory of computation~Type theory}
% \ccsdesc[300]{Theory of computation~Type structures}
% \ccsdesc[300]{Theory of computation~Constructive mathematics}

\keywords{Agda, constructive mathematics, dependent types, equational logic, formalization of mathematics, model theory, type theory, universal algebra}

% \category{} %optional, e.g. invited paper

% \relatedversion{hosted on arXiv}
% \relatedversiondetails[linktext={http://arxiv.org/a/demeo\_w\_1}]{Part 2, Part 3}{http://arxiv.org/a/demeo_w_1}

% \supplement{~\\ \textit{Documentation}: \ualibdotorg}%
% \supplementdetails{Software}{https://gitlab.com/ualib/ualib.gitlab.io.git}

\nolinenumbers %uncomment to disable line numbering

\hideLIPIcs  %uncomment to remove references to LIPIcs series (logo, DOI, ...), e.g. when preparing a pre-final version to be uploaded to arXiv or another public repository

%Editor-only macros:: begin (do not touch as author)%%%%%%%%%%%%%%%%%%%%%%%%%%%%%%%%%%
\EventEditors{}
\EventNoEds{2}
\EventLongTitle{}
\EventShortTitle{}
\EventAcronym{}
\EventYear{2021}
\EventDate{May 21, 2021}
\EventLocation{}
\EventLogo{}
\SeriesVolume{0}
\ArticleNo{0}
%%%%%%%%%%%%%%%%%%%%%%%%%%%%%%%%%%%%%%%%%%%%%%%%%%%%%%




\begin{document}

\maketitle

\section{Introduction}\label{sec:introduction}

This is essentially a glossary of the most basic and important model
theory concepts along with some elementary facts about them.

% \begin{center}\rule{0.5\linewidth}{\linethickness}\end{center}

\section{Signatures, Languages, Terms, Formulas, Sentences}\label{signatures-languages-terms-formulas-sentences}

\subsection{Signatures}\label{signatures}

\begin{itemize}
% %\tightlist
\item
  A \textbf{signature}
  \(\sigma = (\mathbf{C}, \mathbf{F}, \mathbf{R}, \sigma')\) consists of
  three (possibly empty) sets \(\mathbf{C}\), \(\mathbf{F}\), and
  \(\mathbf{R}\) of \emph{constant}, \emph{function}, and
  \emph{relation} symbols (resp.), along with a function
  \(\sigma': \mathbf{C} + \mathbf {F} + \mathbf{R} \to \mathbb N\) that
  assigns an \emph{arity} to each symbol.
\end{itemize}

% \begin{center}\rule{0.5\linewidth}{\linethickness}\end{center}

\subsection{Languages}\label{languages}

\begin{itemize}
%\tightlist
\item
  The \textbf{language} \(L = L(\sigma)\) of signature \(\sigma\) is a
  certain collection of strings of symbols from the following
  \textbf{alphabet} of symbols:
\item
  \textbf{logical symbols}

  \begin{itemize}
  %\tightlist
  \item
    \emph{logical connectives}: \(\neg\) , \(\wedge\) , \(\vee\) (resp.
    ``negation,'' ``conjunction,'' ``disjunction),
  \item
    \emph{existential quantifier:} \(\exists\)
  \item
    \emph{equality}: \(=\)
  \end{itemize}
\item
  \textbf{variables} (countably many)
\item
  \textbf{non-logical symbols} from \(\sigma\) (the constant, function,
  and relation symbols)
\item
  \textbf{parentheses} ( and )
\end{itemize}

To specify which strings of symbols belong to \(L\), we need to define
the \emph{terms} and \emph{formulas} of \(\sigma\). Eventually, we will
define \(L\) to be \emph{the set of all \(\sigma\)-formulas.}

% \begin{center}\rule{0.5\linewidth}{\linethickness}\end{center}

\subsection{Terms}\label{terms}

The \textbf{terms} of signature \(\sigma\) (aka
\(\sigma\)-\textbf{terms}) are defined recursively as follows: - All
variables are terms. - All constant symbols are terms. - If
\(t_0, \dots, t_{n-1}\) are terms and \(f\in \mathbf F\) with
\(\sigma'(f) = n\), then \(f(t_0,\dots, t_{n-1})\) is a term. - \(t\) is
a terms if it can be obtained in finitely many steps from some
combination of the three items above.

% \begin{center}\rule{0.5\linewidth}{\linethickness}\end{center}

\subsection{Formulas}\label{formulas}

The \textbf{formulas} of \(\sigma\) (aka \(\sigma\)-formulas) are
defined recursively as follows: - If \(t_1\) and \(t_2\) are
\(\sigma\)-terms, then \(t_1 = t_2\) is a \(\sigma\)-formula. - If
\(t_0,\dots, t_{n-1}\) are \(\sigma\)-terms and \(R \in \mathbf R\) with
\(\sigma'(R)=n\), then \(R(t_0,\dots, t_{n-1})\) is a
\(\sigma\)-formula. - If \(\varphi\) and \(\psi\) are
\(\sigma\)-formulas and \(x\) is a variable, then \(\neg \varphi\),
\(\varphi \wedge \psi\), and \(\exists x \varphi\) are formulas too. -
\(\varphi\) is a fornıula if it can be obtained in finitely many steps
from some combination of the three items above.

Finally, we can define + The \textbf{language} \(L = L(\sigma)\) is the
set of all \(\sigma\)-formulas.

% \begin{center}\rule{0.5\linewidth}{\linethickness}\end{center}

\subsection{Sentences}\label{sentences}

\begin{itemize}
%\tightlist
\item
  A term \(t\) is said to be \textbf{constant} (or \textbf{closed}) if
  it contains no variables.
\item
  A formula \(\varphi\) is called a \textbf{sentence} (or \textbf{closed
  formula}) if it contains \emph{no \textbf{free} variables}; that is,
  all variables appearing in \(\varphi\) are bound;
\item
  \(L_0 :=\) all sentences in the language \(L\) (aka
  ``\(L\)-sentences'');
\item
  An \textbf{atomic} \(L\)-formula has one of the following two forms:
\item
  \(s = t\), where \(s\) and \(t\) are \(L\)-terms;
\item
  \(R(t_0, \dots, t_{n-1})\), where \(R\) is an relation symbol in \(L\)
  and \(t_i\) are \(L\)-terms;
\item
  \(\mathbf{at}_L\) (or just \(\mathbf{at}\) when the context makes
  \(L\) clear) is the class of all atomic \(L\)-formulas.
\item
  An \textbf{atomic} \(L\)-sentence is either an equation of constant
  terms or a relational sentence, \(R(t_0, \dots, t_{n-1})\), where all
  \(t_i\) are closed terms;
\item
  A \textbf{literal} \(L\)-formula (or, \(L\)-\textbf{literal}) is an
  atomic or negated atomic \(L\)-formula;
\item
  \(\mathbf{lt}_L:=\) the set of all \(L\)-\textbf{literals}; that is,
  \(\mathbf{at}_L \cup \{\neg \varphi : \varphi \in \mathbf{at}_L\}\);
\item
  \(\mathbf{cl}_L:=\) the set of all \textbf{closed}
  \(L\)-\textbf{literals} (literal \(L\)-sentences; that is, literals
  without free variables).
\end{itemize}

\textbf{Remarks.} + Every constant symbol is a constant term. + An
atomic sentence contains no variables at all. + \emph{Languages without
constant symbols have no atomic sentences.} + Every language comes
equipped with a countable supply of variables, so the cardinality of
\(L\) is
\(|L| = \max \{\aleph_0, |\mathbf C \cup \mathbf F \cup \mathbf R|\}\).

% \begin{center}\rule{0.5\linewidth}{\linethickness}\end{center}

\section{Boolean comibinations and quantifier-free
formulas}\label{boolean-comibinations-and-quantifier-free-formulas}

Let \(\Sigma\) be a set of formulas.

\begin{itemize}
\item
  A \textbf{boolean combination} of formulas from \(\Sigma\) is obtained
  by connecting formulas from \(\Sigma\) using only the logical
  connectives; i.e., only \(\vee\), \(\wedge\), \(\neg\).
\item
  A \textbf{positive boolean combination} of formulas from \(\Sigma\) is
  obtained by connecting formulas from \(\Sigma\) with only \(\wedge\)
  and \(\vee\).
\item
  The \textbf{boolean closure} of \(\Sigma\) is the set
  \(\tilde{\Sigma}\) of all boolean combinations of formulas from
  \(\Sigma\).
\item
  A \textbf{positive} formula is obtained from atomic formulas using
  only \(\wedge\), \(\vee\), \(\exists\), \(\forall\).\\
  The class of all positive formulas (of all possible languages) is
  denoted by \(\boldsymbol{+}\).
\item
  A \textbf{negative} formula is a negated positive formula. The class
  of all such is denoted by \(\boldsymbol{-}\).
\item
  A \textbf{quantifier-free} formula is one that contains no
  quantifiers; we assume \(\top\) and \(\perp\) are quantifier-free.\\
  The class of all quantilier-free formulas (of arbitrary signature) is
  denoted by \(\mathbf{qf}\).
\end{itemize}

\textbf{Remarks.} + In the definition of boolean combinations, we could
allow \(\to\) and \(\leftrightarrow\); we could do without \(\vee\). +
\(\mathbf{qf}\) is the class of all boolean combinations of atomic
formulas.

% \begin{center}\rule{0.5\linewidth}{\linethickness}\end{center}

\section{Expansion by Constants, Validity,
Truth}\label{expansion-by-constants-validity-truth}

Fix a signature
\(\sigma = (\mathbf{C}, \mathbf{F}, \mathbf{R}, \sigma')\) and a
language \(L = L(\sigma)\) (i.e., \(L\) is a language of signature
\(\sigma\)). Let \(\mathcal M = \langle M, \dots\rangle\) and
\(\mathcal N = \langle N, \dots\rangle\) be \(L\)-structures, let
\(\mathbf x = (x_0, x_{1}, \dots)\) be a tuple of variables, and let
\(\varphi = \varphi(\mathbf x)\) be an \(L\)-formula.

% \begin{center}\rule{0.5\linewidth}{\linethickness}\end{center}

\subsection{Expansion by Constants}\label{expansion-by-constants}

\begin{itemize}
%\tightlist
\item
  A \textbf{new constant} (\textbf{symbol}) for \(L\) is any symbol not
  occuring in the alphabet of \(L\).
\item
  \(L(C)\) is the \textbf{expansion} of \(L\) by new constant symbols
  \(C\), and is defined to be the (uniquely determined) language of
  signature \((\mathbf C \cup C, \mathbf F, \mathbf R, \sigma')\).
\item
  \(\Delta(C)\) is the \textbf{expansion} of \(\Delta\) by new constants
  symbols \(C\) (not occuring in \(\Delta\)) and is defined to be the
  class of all the formulas obtained from formulas
  \(\varphi \in \Delta\) upon substituting (at will) elements from \(C\)
  for variables in \(\varphi\). (``At will'' indicates that
  \(\Delta\subseteq \Delta(C)\).)
\item
  \(\mathbf{lt}_{L(M)}:=\) the set of all atomic and negated atomic
  \(L(M)\)-formulas.
\item
  \(\mathbf{cl}_{L(M)}:=\) the set of all atomic and negated atomic
  \(L(M)\)-setnences.
\end{itemize}

% \begin{center}\rule{0.5\linewidth}{\linethickness}\end{center}

\subsection{Validity and Truth}\label{validity-and-truth}

\begin{itemize}
%\tightlist
\item
  \(\mathcal M \vDash \varphi\) means that \(\varphi\) is \textbf{valid}
  in \(\mathcal M\) which means that for every tuple
  \(\mathbf a = (a_0, a_1, \dots )\) from \(M\) (that is at least as
  long as \(\mathbf x\)) the \(L\)-sentence \(\varphi(\mathbf a)\) is
  \emph{true in \(\mathcal M\)}.
\end{itemize}

\emph{What exactly do we mean by ``\(\varphi(\mathbf a)\) is true in
\(\mathcal M\)?''}

Intuitively, for each \(i\) we subsitute the element \(a_i\) for the
variable \(x_i\) in the formula \(\varphi(\mathbf x)\), which yields a
sentence \(\varphi(\mathbf a)\) that is ``decidable'' in \(\mathcal M\).
That is, there is a finite procedure by which we can determine whether
or not \(\varphi(\mathbf a)\) holds (or is ``true'') in \(\mathcal M\).

Formally, however, we may follow a more careful procedure for judging
the truth of a given formula \(\varphi\) in a given structure
\(\mathcal M = \langle M, \dots\rangle\). This is the simple matter of
how syntactically to denote interpretation of variables. But as Hodges
puts it, this is ``one of the more irksome parts of model theory.''

The issue is explained clearly in Section 1.4 of Hodges, ``Model
Theory,'' so we defer to that presentation:

\begin{itemize}
%\tightlist
\item
  The conventions for interpreting variables are one of the more irksome
  parts of model theory. We can avoid them, at a price. Instead of
  interpreting a variable as a name of the element \(b\), we can add a
  new constant for \(b\) to the signature. The price we pay is that the
  language changes every time another element is named. When constants
  are added to a signature, the new constants and the elements they name
  are called parameters.
\end{itemize}

Suppose for example that \(\mathcal A\) is an \(L\)-structure,
\(\mathbf a\) is a sequence of elements of \(\mathcal A\), and we want
to name the elements in \(\mathbf a\). Then we choose a sequence
\(\mathbf c\) of distinct new constant symbols, of the same length as
\(\mathbf a\), and we form the signature \(L(\mathbf c)\) by adding the
constants \(\mathbf c\) to \(L\). Then \((\mathcal A, \mathbf a)\) is an
\(L(\mathbf c)\)-structure, and each element \(a_i\) is
\(c_i^{(\mathcal A, \mathbf a)}\).

Likewise if \(\mathcal B\) is another \(L\)-structure and \(\mathbf b\)
a sequence of elements of \(\mathcal B\) of the same length as
\(\mathbf c\), then there is an \(L(\mathbf c)\)-structure
\((\mathcal B, \mathbf b)\) in which these same constants \(c_i\) name
the elements of \(\mathbf b\). The next lemma is about this situation.
It comes straight out of the definitions, and it is often used silently.

\begin{lemma}[1.4.1 of \cite{Hodges:1993}] Let \(\mathcal A\), \(\mathcal B\) be
\(L\)-structures and suppose \((\mathcal A, \mathbf a)\),
\((\mathcal B, \mathbf b)\) are \(L(\mathbf c)\)-structures. Then a
homomorphism
\(f \colon (\mathcal A, \mathbf a) \to (\mathcal B, \mathbf b)\) is the
same thing as a homomorphism \(f \colon \mathcal A \to \mathcal B\) such
that \(f[\mathbf a] = \mathbf b\).
\end{lemma}

In the situation above, if \(t(\mathbf x)\) is a term of \(L\), then
\(t^{\mathcal A}(\mathbf a)\) and
\(t(\mathbf c)^{(\mathcal A, \mathbf a)}\) are the same element; and if
\(\varphi(\mathbf x)\) is an atomic formula then
\(\mathcal A \vDash \varphi_{\mathbf x}(\mathbf a) \; \Leftrightarrow \; (\mathcal A, \mathbf c) \vDash \varphi_{\mathbf x}(\mathbf c)\).

\textbf{Notation} (used above) \(f[\mathbf a]\) is shorthand for
\((f(a_0), f(a_1), \dots)\) and \(\varphi_{\mathbf x}(\mathbf a)\) is
shorthand for \([a_0/x_0, a_1/x_1, \dots]\varphi(x_0, x_1, \dots)\), the
sentence obtained from \(\varphi\) upon substituting \(a_i\) for
\(x_i\), for each \(i\).

% \begin{center}\rule{0.5\linewidth}{\linethickness}\end{center}

\section{Models, Theories, Diagrams}\label{models-theories-diagrams}

Let \(\varphi \in L_0\), \(\Sigma\subseteq L_0\), and let
\(\mathcal M = \langle M, \dots\rangle\) and
\(\mathcal N = \langle N, \dots\rangle\) be \(L\)-structures. Let
\(\Delta\) be an arbitrary class of formulas (not necessarily from
\(L\)).

% \begin{center}\rule{0.5\linewidth}{\linethickness}\end{center}

\subsection{Models}\label{models}

\begin{itemize}
%\tightlist
\item
  If \(M\neq \emptyset\) and \(\mathcal M \vDash \Sigma\), then
  \(\mathcal M\) is a \textbf{model} of \(\Sigma\); we also say
  ``\(\mathcal M\) \emph{models} \(\Sigma\).''
\item
  \(\operatorname{Mod}_L \Sigma :=\) the class of \(L\)-structures that
  model \(\Sigma\).
\item
  \(\operatorname{Mod}_L \emptyset :=\) the class of all nonempty
  \(L\)-structures.
\item
  \(\Sigma\) \textbf{entails} \(\varphi\), denoted
  \(\Sigma \vdash \varphi\), if every model of \(\Sigma\) also models
  \(\varphi\).
\item
  \(\varphi\) is a \textbf{logical consequence} of \(\Sigma\) means
  \(\Sigma\) entails \(\varphi\).
\item
  The \textbf{deductive closure} of \(\Sigma\) is the set
  \(\Sigma^{\vdash} = \{\varphi \in L_0 : \Sigma \vdash \varphi\}\) of
  logical consequences of \(\Sigma\).
\item
  \(\Sigma\) \textbf{deductively closed} if
  \(\Sigma^\vdash \subseteq \Sigma\).
\item
  \(\Sigma_0, \Sigma_1 \subseteq L_0\) are
  \(\Sigma\)-\textbf{equivalent} if
  \((\Sigma \cup \Sigma_0)^\vdash = (\Sigma \cup \Sigma_1)^\vdash\).
\item
  \textbf{logically equivalent} means \(\emptyset\)-equivalent.
\item
  A \textbf{contradiction} is an \(L\)-sentence of the form
  \(\varphi \wedge \neg \varphi\).
\item
  \(\Sigma\) is \textbf{consistent} if \(\Sigma^\vdash\) contains no
  contradictions; otherwise, \(\Sigma\) is \textbf{inconsistent}.\\
  \textbf{Remark.} No model satisfies a contradiction, so the deductive
  closure of a contradiction is the set \(L_0\) of all \(L\)-sentences,
  and \(L_0\) is the only deductively closed inconsistent set of
  \(L\)-sentences.
\end{itemize}

% \begin{center}\rule{0.5\linewidth}{\linethickness}\end{center}

\subsection{Theories}\label{theories}

\begin{itemize}
\item
  An \(L\)-\textbf{theory} is a \emph{consistent} and \emph{deductively
  closed} set of \(L\)-sentences.
\item
  The \textbf{cardinality} or \textbf{power} of an \(L\)-theory \(T\) is
  denoted \(|T|\) and defined to be the cardinality of \(L\).
\item
  \(T_\Delta = (T \cap \Delta)^\vdash\) is the \(\Delta\)-\textbf{part}
  of the \(L\)-theory \(T\) (here \(\Delta\) is an arbibtrary class of
  formulas).
\item
  \(\bfall\) is the class of formulas in which \(\exists\) does not
  appear; \(T_{\bfall} = (T \cap \bfall)^\vdash\) is the \emph{universal
  part} of \(T\).
\item
  \(\operatorname{Th}_\Delta \mathcal M := \{\varphi \in L_0 : \varphi \in \Delta,\; \mathcal M \vDash \varphi\}=\)
  all \(L\)-sentences in \(\Delta\) that are true in \(\mathcal M\).
\item
  \(\operatorname{Th} \mathcal M := \operatorname{Th}_{L_0} \mathcal M =\)
  all \(L\)-sentences that are true in \(\mathcal M\).
\item
  An \(L\)-theory \(T\) is \textbf{complete} if for all
  \(\varphi\in L_0\), either \(\varphi \in T\) or
  \(\neg\varphi\in T\).\\
  \textbf{Lemma} (\cite[3.5.1]{Rothmaler:2000}). If \(T\) is an \(L\)-theory, the following are
  equivalent:
\end{itemize}

\begin{enumerate}
\def\labelenumi{\arabic{enumi}.}
%\tightlist
\item
  \(T\) is complete.
\item
  \(T\) is a maximal \(L\)-theory.
\item
  \(T\) is a maximal consistent set of \(L\)-sentences.
\item
  \(T= \operatorname{Th}\mathcal M\) for all \(\mathcal M \vDash T\).
\item
  \(T= \operatorname{Th}\mathcal M\) for some \(\mathcal M \vDash T\).
\end{enumerate}

\begin{itemize}
%\tightlist
\item
  \textbf{Examples.}
\item
  \(T^\infty\) is the theory of the class of all infinite models of
  \(T\).
\item
  \(T_=\) is the \textbf{theory of pure identity}, which is the
  \(L_=\)-theory of all sets (regarded as \(L_=\)-structures).
\end{itemize}

% \begin{center}\rule{0.5\linewidth}{\linethickness}\end{center}

\subsection{Diagrams}\label{diagrams}

\begin{itemize}
%\tightlist
\item
  The \textbf{diagram} of \(\mathcal M\) is the set
  \(D(\mathcal M) := \mathbf{cl}_{L(M)}\) of all atomic and negated
  atomic \(L(M)\)-sentences;
\item
  \(\mathcal M \Rrightarrow_{\Delta} \mathcal N\) means
  \(\mathcal M \vDash \varphi\) implies \(\mathcal N \vDash \varphi\)
  for all \(\varphi \in \Delta \cap L_0\).
\item
  \(\mathcal M \Rrightarrow \mathcal N\) means
  \(\mathcal M \Rrightarrow_{L} \mathcal N\).
\item
  \(\mathcal M \equiv \mathcal N\) means
  \(\mathcal M \Rrightarrow \mathcal N\) and
  \(\mathcal M \Lleftarrow \mathcal N\) hold, and this is equivalent to
  \(\operatorname{Th} \mathcal M = \operatorname{Th} \mathcal N\).\\
  We call \(\mathcal M\) and \(\mathcal N\) \textbf{elementarily
  equivalent} in this case.
\item
  \(f \colon \mathcal M \hookrightarrow \mathcal N\) means \(f\) is an
  \(L\)-structure-monomorphism from \(\mathcal M\) to \(\mathcal N\).
\item
  \(f \colon \mathcal M \stackrel{\Delta}{\longrightarrow} \mathcal N\)
  means all \(L\)-formulas in \(\Delta\) are preserved by \(f\). That
  is,\\
  \(\mathcal M \vDash \varphi(\mathbf a)\) implies
  \(\mathcal N \vDash \varphi(f[\mathbf a])\), for all
  \(\varphi \in \Delta\cap L\) and all tuples \(\mathbf a\) from \(M\).
\item
  \(f \colon \mathcal M \stackrel{\equiv}{\hookrightarrow} \mathcal N\)
  means
  \(f \colon \mathcal M \stackrel{L}{\longrightarrow} \mathcal N\).
\end{itemize}

% \begin{center}\rule{0.5\linewidth}{\linethickness}\end{center}

\subsection{Some Facts}\label{some-facts}

Let \(\mathcal M\) and \(\mathcal N\) be \(L\)-structures and let
\(\Delta\) be a set of \(L\)-formulas. 1.
\(f \colon \mathcal M \stackrel{\mathbf{at}}{\longrightarrow} \mathcal N\)iff
\(f \colon \mathcal M \rightarrow \mathcal N\) 2.
\(f \colon \mathcal M \stackrel{\Delta}{\longrightarrow} \mathcal N\)
iff \((\mathcal M, M) \Rrightarrow_{\Delta(M)} (\mathcal N, f[M])\). 3.
If \(f \colon \mathcal M \stackrel{\Delta}{\longrightarrow} \mathcal N\)
and \(\Delta\) contains \(\mathbf{at}\) and all negations of unnested
relational atomic formulas, then \(f\) is a strong homomorphism. (The
converse is not true.) 4. If
\(f \colon \mathcal M \stackrel{\Delta}{\longrightarrow} \mathcal N\)
and \(\Delta\) is closed under negation, then
\(\mathcal M \vDash \varphi(\mathbf a)\) implies
\(\mathcal N \vDash \varphi(f[\mathbf a])\), for all
\(\varphi \in \Delta\) and tuples \(\mathbf a\) from \(M\). 5.
\(f \colon M \to N\) is injective iff
\(f \colon \mathcal M \stackrel{\Delta}{\longrightarrow} \mathcal N\)
for the set \(\Delta = \{x \neq y\}\). 6. If \(\Delta \subseteq L_0\),
then
\(f \colon \mathcal M \stackrel{\Delta}{\longrightarrow} \mathcal N\)
iff \(\mathcal M \Rrightarrow_{\Delta} \mathcal N\) and
\(f \colon M \to N\). 7. If \(\Delta \subseteq L_0\) and \(\Delta\) is
closed under negation, then
\(\mathcal M \Rrightarrow_\Delta \mathcal N\) implies
\(\mathcal M \equiv_\Delta \mathcal N\).

% \begin{center}\rule{0.5\linewidth}{\linethickness}\end{center}

\subsection{The Lemma on Constants}\label{the-lemma-on-constants}

Above we remarked that if \((\mathcal A, \mathbf a)\) is an
\(L(\mathbf c)\)-structure with \(\mathcal A\) an \(L\)-structure, then
for every atomic formula \(\varphi\) of \(L\),
\(\mathcal A \vDash \varphi(\mathbf a)\) if and only if
\((\mathcal A, \mathbf a) \vDash \varphi(\mathbf c)\).

\begin{lemma}[2.3.2 of \cite{Hodges:1993}] Let \(L\) be a language, \(T\) a
theory in \(L\) and \(\varphi(\mathbf x)\) a formula in \(L\). Let
\(\mathbf c\) be a sequence of distinct constants that are not in \(L\).
Then \(T \vdash \varphi(\mathbf c)\) if and only if
\(T \vdash \forall \mathbf x\, \varphi\).
\end{lemma}
% \begin{center}\rule{0.5\linewidth}{\linethickness}\end{center}

\subsection{The Diagram Lemma}\label{the-diagram-lemma}

\begin{lemma}[6.1.2 of \cite{Rothmaler:2000}] Let \(\mathcal M\) and
\(\mathcal N\) be \(L\)-structures. 1.
\(f \colon \mathcal M \hookrightarrow \mathcal N\)
\(\; \Leftrightarrow \;\)
\(f \colon \mathcal M \stackrel{\mathbf{qf}}{\longrightarrow} \mathcal N\)
\(\; \Leftrightarrow \;\) \((\mathcal N, f[M]) \vDash D(\mathcal M)\).\\
In particular,
\(f \colon \mathcal M \stackrel{\equiv}{\hookrightarrow} \mathcal N \; \Rightarrow \; f \colon \mathcal M \hookrightarrow \mathcal N\).

\begin{enumerate}
\def\labelenumi{\arabic{enumi}.}
\setcounter{enumi}{1}
%\tightlist
\item
  \(\mathcal M \hookrightarrow \mathcal N\) \(\; \Leftrightarrow \;\)
  \(\mathcal N\) has an \(L(M)\)-expansion that models
  \(D(\mathcal M)\).
\end{enumerate}
\end{lemma}
% \begin{center}\rule{0.5\linewidth}{\linethickness}\end{center}

\subsection{The Diagram Lemma (ver.
2)}\label{the-diagram-lemma-ver.-2}

Let's consider an alternative version of the Diagram Lemma that makes
the role played by new constants more explicit. For this version, we
will use the following additional notation:

\begin{itemize}
%\tightlist
\item
  \(\mathbf c = (c_0, \dots, c_{n-1})=\) an arbitrary tuple of distinct
  symbols not appearing in \(L\);
\item
  \(\mathbf a = (a_0, \dots, a_{n-1}) \in M^n\),
  \(\mathbf b = (b_0, \dots, b_{n-1}) \in N^n\);
\item
  \((\mathcal M, \mathbf c)\) and \((\mathcal N, \mathbf c)\) are
  \(L(\mathbf c)\)-structures, where \(c_i^{\mathcal M} = a_i\) and
  \(c_i^{\mathcal N} = b_i\) are the interpretations in \(\mathcal M\)
  and \(\mathcal N\) of the new constant symbols;
\item
  \(\langle \mathbf a \rangle\) is the substructure of \(\mathcal M\)
  generated by the elements of the tuple \(\mathbf a\);
\item
  \(f[\mathbf a]:=(f(a_0), \dots, f(a_{n-1}))\).
\end{itemize}

\begin{lemma}[1.4.2 of \cite{Hodges:1993}] The following are equivalent:\\
  \begin{enumerate}
  \item For every atomic sentence \(\varphi(\mathbf c)\) of \(L(\mathbf c)\),
if \((\mathcal M, \mathbf c) \vDash \varphi(\mathbf c)\) then
\((\mathcal N, \mathbf c) \vDash \varphi(\mathbf c)\). 
\item There is a
homomorphism \(f \colon \langle \mathbf a \rangle \to \mathcal N\) such
that \(f[\mathbf a] = \mathbf b\).\\
\item The homomorphism in 2 is unique, if it exists, and it is an embedding
if and only if:\\
for every atomic sentence \(\varphi\) of \(L(\mathbf c)\), we have
\((\mathcal M, \mathbf c) \vDash \varphi \; \Leftrightarrow \; (\mathcal N, \mathbf c) \vDash \varphi\).
  \end{enumerate}
\end{lemma}
% \begin{center}\rule{0.5\linewidth}{\linethickness}\end{center}

\section{Elementary equivalence}\label{elementary-equivalence}

\subsection{Isomorphic structures are elementarily
equivalent}\label{isomorphic-structures-are-elementarily-equivalent}

\textbf{Proposition}~\cite[6.1.3]{Rothmaler:2000} If
\(f \colon \mathcal M \cong \mathcal N\), then
\(f \colon \mathcal M \stackrel{\equiv}{\hookrightarrow} \mathcal N\),
hence also \(\mathcal M \equiv \mathcal N\).

Proof of the first implication is on page 70 of the text.

The second implication is
\(f \colon \mathcal M \stackrel{\equiv}{\hookrightarrow} \mathcal N\)
implies \(\mathcal M \equiv \mathcal N\). That's true because, if
\(f \colon \mathcal M \stackrel{\equiv}{\hookrightarrow} \mathcal N\),
then we have not only \(\varphi \in \operatorname{Th} \mathcal M\)
implies \(\varphi \in \operatorname{Th}\mathcal N\), but also
\(\neg \varphi \in \operatorname{Th} \mathcal M\) implies
\(\neg \varphi \in \operatorname{Th}\mathcal N\). From the former,
\(\mathcal M \Rrightarrow \mathcal N\). From the latter,
\(\mathcal M \Lleftarrow \mathcal N\). Thus,
\(\mathcal M \equiv \mathcal N\).

% \begin{center}\rule{0.5\linewidth}{\linethickness}\end{center}

The converse of 6.1.3 holds if and only if the structures involved are
finite, as the next proposition shows.

\textbf{Proposition}~\cite[8.1.1]{Rothmaler:2000} Let \(\mathcal M\) be an
\(L\)-structure. The following are equivalent: 1.
\(\mathcal N \equiv \mathcal M\) implies \(\mathcal N \cong \mathcal M\)
for any \(L\)-structure \(\mathcal N\). 2. \(\mathcal M\) is finite.

% \begin{center}\rule{0.5\linewidth}{\linethickness}\end{center}

\subsection{All models of a complete theory are elementarily
equivalent}\label{all-models-of-a-complete-theory-are-elementarily-equivalent}

\textbf{Proposition} {[}8.1.2. Rothmaler{]} A theory is complete iff its
models are elementarily equivalent.

\textbf{Corollary} {[}8.1.3. Rothmaler{]} A complete theory has a finite
model iff it has only one model (up to isomorphism).

% \begin{center}\rule{0.5\linewidth}{\linethickness}\end{center}

\section{Elementary maps}\label{elementary-maps}

Let \(\mathcal M\) and \(\mathcal N\) be \(L\)-structures. A map
\(f \colon M \to N\) is called \textbf{elementary} if
\(f \colon \mathcal M \stackrel{\equiv}{\hookrightarrow} \mathcal N\).
The structure \(\mathcal M\) is \textbf{elementarily embeddable} in
\(\mathcal N\), in symbols
\(\mathcal M \stackrel{\equiv}{\hookrightarrow} \mathcal N\), if there
is an elementary map
\(f \colon \mathcal M \stackrel{\equiv}{\hookrightarrow} \mathcal N\).

\textbf{Remarks.}\\
1. If
\(f \colon \mathcal M \stackrel{\equiv}{\hookrightarrow} \mathcal N\),
then \(\mathcal M \equiv \mathcal N\). 2. While elementary equivalence
is weaker than isomorphism, every elementary map is automatically an
isomorphic embedding (by Lemma 6.1.2(1)). Therefore elementary maps are
also called \textbf{elementary embeddings}, and the notation
\(f \colon \mathcal M \stackrel{\equiv}{\hookrightarrow} \mathcal N\) is
justified. 3. The converse is not true in general, unless the embedding
is surjective (i.e., an isomorphism), since 4. Proposition 6.1.3 says
that every isomorphism \(f \colon \mathcal M \cong \mathcal N\) is an
elementary map.

% \begin{center}\rule{0.5\linewidth}{\linethickness}\end{center}

\subsection{Elementary Diagram Lemma}\label{elementary-diagram-lemma}

The \textbf{elementary diagram} of an \(L\)-structure \(\mathcal M\) is
the complete \(L(M)\)-theory \(\operatorname{Th}(\mathcal M, M)\).

\begin{lemma}[8.2.1 of \cite{Rothmaler:2000}]
 Let \(\mathcal M\) and
\(\mathcal N\) be \(L\)-structures and \(f \colon M \rightarrow N\). 1.
\(f\) is elementary \(\; \Leftrightarrow \;\)
\((\mathcal M, M) \equiv (\mathcal N, f[M])\) \(\; \Leftrightarrow \;\)
\((\mathcal N, f[M]) \vDash Th(\mathcal M, M)\). 2.
\(\mathcal M \stackrel{\equiv}{\hookrightarrow} \mathcal N\)
\(\; \Leftrightarrow \;\) iff \(\mathcal N\) has an expansion that is a
model of \(\operatorname{Th}(\mathcal M, M)\).
\end{lemma}
%%%%%%%%%%%%%%%%%%%%%%%%%%%%%%%%%%%%%%%%%%%%%%%%%%%%%%%%%%%%%%%%%%%%%%%%%%%%%%%%%%%%%%%%%% 
%%%%%%%%%%%%%%%%%%%%%%%%%%%%%%%%%%%%%%%%%%%%%%%%%%%%%%%%%%%%%%%%%%%%%%%%%%%%%%%%%%%%%%%%%%

%% Bibliography
\bibliographystyle{plainurl}% the mandatory bibstyle

\bibliography{mtrefs}


\end{document}