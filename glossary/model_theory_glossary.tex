% -*- TeX-master: "ualib-part1.tex" -*-
%%% Local Variables: 
%%% mode: latex
%%% TeX-engine: 'xetex
%%% End:
\documentclass[a4paper,UKenglish,cleveref,autoref,thm-restate,12pt]{lipics-v2021-wjd}
\usepackage{wjd}
% \usepackage{fontspec}
\usepackage{comment}
\usepackage{proof-dashed}
%\usepackage{unixode}
\usepackage[wjdsimple,links]{agda}
\usepackage{hyperref}

\hypersetup{
%     bookmarks=true,         % show bookmarks bar?
%     unicode=true,          % non-Latin characters in Acrobat’s bookmarks
%     pdftoolbar=true,        % show Acrobat’s toolbar?
%     pdfmenubar=true,        % show Acrobat’s menu?
%     pdffitwindow=false,     % window fit to page when opened
% %    pdfstartview={FitW},    % fits the width of the page to the window
    pdftitle={Agda UALib, Part 1: Foundation},    % title
    pdfauthor={William DeMeo},     % author
    pdfsubject={Agda Universal Algebra Library},   % subject of the document
%     pdfcreator={XeLaTeX with hyperref},
%     pdfproducer={},  % producer of the document
%     pdfkeywords={Universal algebra} {Equational logic} {Martin-Löf Type Theory} {Birkhoff’s HSP Theorem} {Formalization of mathematics} {Agda} {Proof assistant},
%     pdfnewwindow=true,      % links in new window
%     colorlinks=true,       % false: boxed links; true: colored links
%     linkcolor=red,          % color of internal links
%     citecolor=green,        % color of links to bibliography
%     filecolor=magenta,      % color of file links
%     urlcolor=cyan           % color of external links
}
\newif\ifnonbold % comment this line to restore bold universe symbols

% \usepackage[x11names, rgb]{xcolor}
\usepackage{tikz}
\usetikzlibrary{snakes,arrows,shapes}
\usepackage{amsmath}
% \documentclass{amsart}
% \usepackage{amsmath}

\newcommand{\inr}{\operatorname{inr}}
\newcommand{\inl}{\operatorname{inl}}
\newcommand{\DLO}{\operatorname{DLO}}
\newcommand{\<}{\langle}
\renewcommand{\>}{\rangle}
\renewcommand\S{\operatorname{S}}
\newcommand\B{\operatorname{B}}
\renewcommand\L{\operatorname{L}}
\newcommand\cM{\mathcal{M}}
\newcommand\cN{\mathcal{N}}
\newcommand\CD{\mathcal{D}}
\newcommand\N{\mathbb{N}}
\newcommand\Z{\mathbb{Z}}
\newcommand\Th{\operatorname{Th}}
\newcommand\cl{\mathbf{cl}}
\newcommand\lt{\mathbf{lt}}
\newcommand\bfall{\boldsymbol{\forall}}
\newcommand\bfex{\boldsymbol{\exists}}

%%%%%%%%%%%%%% TITLE %%%%%%%%%%%%%%%%%%%%%%%%%%%%%%%%%%%%%%%%%%%%%%%%%%%%%%%%%%%%%%%%
\title{Model Theory Nomenclature}% and quotients}
%The Agda Universal Algebra Library, Part 1: Foundation. Equality, extensionality, truncation, and dependent types for relations and algebras


%%%%%%%%%%%%%% AUTHOR %%%%%%%%%%%%%%%%%%%%%%%%%%%%%%%%%%%%%%%%%%%%%%%%%%%%%%%%%%%%%%%%
\author{William DeMeo}
       {Department of Algebra, Charles University in Prague \and \url{https://williamdemeo.gitlab.io}}
       {williamdemeo@gmail.com}{https://orcid.org/0000-0003-1832-5690}{}
%\authorrunning{W.~J.~DeMeo}

\copyrightfooter

% \pagestyle{fancy}
% \renewcommand{\sectionmark}[1]{\markboth{#1}{}}
% \fancyhf{}
% \fancyhead[ER]{\sffamily\bfseries \leftmark}
% \fancyhead[OL]{\sffamily\bfseries Agda UALib, Part 1: Foundation}
% \fancyhead[EL,OR]{\sffamily\bfseries \thepage}

% \ccsdesc[500]{Theory of computation~Logic and verification}
% \ccsdesc[300]{Computing methodologies~Representation of mathematical objects}
% \ccsdesc[300]{Theory of computation~Type theory}
% \ccsdesc[300]{Theory of computation~Type structures}
% \ccsdesc[300]{Theory of computation~Constructive mathematics}

\keywords{Agda, constructive mathematics, dependent types, equational logic, formalization of mathematics, model theory, type theory, universal algebra}

% \category{} %optional, e.g. invited paper

% \relatedversion{hosted on arXiv}
% \relatedversiondetails[linktext={http://arxiv.org/a/demeo\_w\_1}]{Part 2, Part 3}{http://arxiv.org/a/demeo_w_1}

% \supplement{~\\ \textit{Documentation}: \ualibdotorg}%
% \supplementdetails{Software}{https://gitlab.com/ualib/ualib.gitlab.io.git}

\nolinenumbers %uncomment to disable line numbering

\hideLIPIcs  %uncomment to remove references to LIPIcs series (logo, DOI, ...), e.g. when preparing a pre-final version to be uploaded to arXiv or another public repository

%Editor-only macros:: begin (do not touch as author)%%%%%%%%%%%%%%%%%%%%%%%%%%%%%%%%%%
\EventEditors{}
\EventNoEds{2}
\EventLongTitle{}
\EventShortTitle{}
\EventAcronym{}
\EventYear{2021}
\EventDate{May 21, 2021}
\EventLocation{}
\EventLogo{}
\SeriesVolume{0}
\ArticleNo{0}
%%%%%%%%%%%%%%%%%%%%%%%%%%%%%%%%%%%%%%%%%%%%%%%%%%%%%%




\begin{document}

\maketitle

\begin{abstract}
% \section{Introduction}\label{sec:introduction}

This is a tourist's guide to model theory, including the most basic and
important concepts of the theory, along with some elementary facts about them.
To keep the presentation concise, proofs are omitted but may be found in the
references cited.
\end{abstract}
% \begin{center}\rule{0.5\linewidth}{\linethickness}\end{center}

\section{Signatures, Languages, Terms, Formulas, Sentences}\label{signatures-languages-terms-formulas-sentences}

\subsection{Signatures}\label{signatures}

A \defn{signature} 𝑆 = (𝑪, 𝑭, 𝑹, ρ) consists of three (possibly empty) sets 𝑪, 𝑭, and 𝑹 of \emph{constant},
\emph{function}, and \emph{relation} symbols (resp.), along with a function
ρ : 𝑪 + 𝑭 + 𝑹 → ℕ that assigns an \emph{arity} to each symbol.

\subsection{Languages}\label{languages}
The \defn{language} 𝐿 = 𝐿(𝑆) of signature 𝑆 is a certain collection of strings
of symbols from the following \defn{alphabet} of symbols:
\begin{enumerate}
\item  \defn{logical symbols}:
  \begin{enumerate}
  \item \emph{logical connectives}: ¬ , ∧ , ∨ , ⊤ , ⊥ (for ``negation,''
    ``conjunction,'' ``disjunction,'' ``true,'' ``false,'' resp.),
  \item \emph{existential quantifier}: ∃
  \item \emph{equality}: =
  \end{enumerate}
\item \defn{variable symbols} (countably many)
\item \defn{non-logical symbols} from 𝑆 (the constant, function, and relation symbols)
\item \defn{parentheses}: ( , ).
\end{enumerate}
We often denote the (countable) set of variable symbols by 𝑋.
To specify which strings of symbols belong to the language 𝐿 we must define the
\emph{terms} and \emph{formulas} of 𝑆 (called 𝑆-terms and 𝑆-formulas,
respectively). Eventually, we will \emph{define} 𝐿 as follows:
\begin{quote}
\emph{The language 𝐿 is the set of all 𝑆-formulas.}
\end{quote}





\subsection{Terms}\label{terms}
The set \(\operatorname{Term}_S(X)\) of \defn{terms} in
the signature 𝑆 generated by
the set 𝑋 of variable symbols is defined recursively as follows:\footnote{The
  set \(\operatorname{Term}_S(X)\) may be denoted by \(\operatorname{Term}(X)\)
  when the signature is clear from the context, and the members of this set go
  by the following synonymous names: 𝑆-\defn{terms}, 𝐿(𝑆)-terms, or 𝐿-terms.}
\begin{enumerate}
\item All variables symbols (in 𝑋) are 𝑆-terms.
\item All constant symbols (in 𝑪) are 𝑆-terms.
\item If 𝑓 ∈ 𝑭 and 𝑡 is a sequence of 𝑆-terms of length ρ 𝑓, then 𝑓 𝑡 = 𝑓(𝑡₀, 𝑡₁, …) is an 𝑆-term.
\item Any string of symbols that can be obtained in finitely many steps from some
combination of 1, 2, and 3 above is an 𝑆-term.
\end{enumerate}






\subsection{Formulas}\label{formulas}
The \defn{formulas} of 𝑆 (also known as 𝑆-formulas, or 𝐿(𝑆)-formulas, or 𝐿-formulas) are defined recursively as follows:
\begin{enumerate}
\item ⊤ and ⊥ are 𝑆-formulas (the ``trivially true'' and ``trivially false'' formulas).
\item If 𝑠 and 𝑡 are 𝑆-terms, then 𝑠 = 𝑡 is an 𝑆-formula.
\item If 𝑟 ∈ 𝑹 and 𝑡 is a sequence of 𝑆-terms of length ρ 𝑟, then
      𝑟 𝑡 = 𝑟 (𝑡₀, 𝑡₁, … ) is an 𝑆-formula.
\item If φ and ψ are 𝑆-formulas and 𝑥 ∈ 𝑋, then ¬ φ, φ ∧ ψ, and ∃ 𝑥 φ are 𝑆-formulas.
\item Any string of symbols that can be obtained in finitely many steps
      from some combination of 1--4 above is an 𝑆-formula.
\end{enumerate}
The \defn{language of the signature} 𝑆, denoted 𝐿(𝑆) (or just 𝐿), is the set of all 𝑆-formulas.



\subsection{Sentences}\label{sentences}
A variable is \defn{bound} if it is captured by (falls within the scope of) a
quantifier or a λ. For example, x and y are bound and z is free in the following
formulas: (∀ x)(∃ y)(x = z + y), and λ~x~y (x~+~y~+~z) = λ y (z + λ x (x + y)).
\begin{enumerate}
\item
  A term 𝑡 is said to be \defn{constant} (or \defn{closed}) if
  it contains no variables.
\item
  A formula φ is called a \defn{sentence} (or \defn{closed
  formula}) if it contains no \emph{free} variables; that is,
  all variables appearing in φ are bound.
\item
  𝐿₀ := all sentences in the language 𝐿 (aka ``𝐿-sentences'');
\item
  An \defn{atomic} 𝐿-formula has one of the following two forms:
  \begin{enumerate} 
  \item 𝑠 = 𝑡, where 𝑠 and 𝑡 are 𝐿-terms;
  \item 𝑟 𝑡, where 𝑟 ∈ 𝑹 and 𝑡 is a sequence of 𝐿-terms of length ρ 𝑟.
  \end{enumerate}
\item
  \(\mathbf{at}_L\) (or just \(\mathbf{at}\) when the context makes
  𝐿 clear) is the class of all atomic 𝐿-formulas.
\item
  An \defn{atomic} 𝐿-sentence is either an equation of constant
  terms or a relational sentence, \(R(t_0, \dots, t_{n-1})\), where all
  \(t_i\) are closed terms;
\item
  A \defn{literal} 𝐿-formula (or, 𝐿-\defn{literal}) is an
  atomic or negated atomic 𝐿-formula;
\item
  \(\mathbf{lt}_L:=\) the set of all 𝐿-\defn{literals}; that is,
  \(\mathbf{at}_L \cup \{¬ φ : φ \in \mathbf{at}_L\}\);
\item
  \(\mathbf{cl}_L:=\) the set of all \defn{closed}
  𝐿-\defn{literals} (literal 𝐿-sentences; that is, literals
  without free variables).
\end{enumerate}
\begin{remarks*}\
\begin{enumerate}
\item Every constant symbol is a constant term.
\item An atomic sentence contains no variables at all.
\item \emph{Languages without constant symbols have no atomic sentences.}
\item Every language comes equipped with a countable supply of variables, so the
      cardinality of 𝐿 is |𝐿| = max \{ℵ₀, | 𝑪 ∪ 𝑭 ∪ 𝑹 |\}.
\end{enumerate}
\end{remarks*}

% \begin{center}\rule{0.5\linewidth}{\linethickness}\end{center}

\section{Boolean comibinations and quantifier-free formulas}\label{boolean-comibinations-and-quantifier-free-formulas}

Let ℰ be a set of formulas.

\begin{enumerate}
\item
  A \defn{boolean combination} of ℰ is obtained by connecting formulas from ℰ
  using only connectives from \{∧, ∨, ¬\}.
\item
  A \defn{positive boolean combination} of ℰ is obtained by connecting formulas
  from ℰ using only connectives from \{∧, ∨\}.
\item
  The \defn{boolean closure} of ℰ is the set \(\tilde{ℰ}\) of all boolean combinations of
  formulas from ℰ.
\item
  A \defn{positive} formula is obtained from atomic formulas using
  only connectives or quantifiers from \(\{∧, ∨, ∃, ∀\}\).
\item The class of all positive formulas (of all possible languages) is denoted by \(\boldsymbol{+}\).
\item A \defn{negative} formula is a negated positive formula, and the class
  of such is denoted by \(\boldsymbol{-}\).
\item A \defn{quantifier-free} formula is one with no quantifiers.
\item The class of all quantilier-free formulas (of arbitrary signature) is
  denoted by \(\mathbf{qf}\).
\end{enumerate}
\begin{remarks*}\
\begin{enumerate}
\item In the definition of boolean combinations, we could allow → and ←; we
  could do without ∨.
\item Formulas in \(\mathbf{qf}\) have no occurrences of ∃ or ∀; ⊤ and ⊥ are
  such formulas, so ⊤, ⊥ ∈ \(\mathbf{qf}\).
\item \(\mathbf{qf}\) is the class of all boolean combinations of atomic formulas.
\end{enumerate}
\end{remarks*}

\section{Expansion by Constants, Validity, Truth}\label{expansion-by-constants-validity-truth}

Fix a signature 𝑆 = (𝑪, 𝑭, 𝑹, ρ) and a let
\begin{itemize}
\item 
  𝐿 = 𝐿(𝑆), the language of 𝑆;
\item 
  Δ = an arbitrary class of formulas (not necessarily from 𝐿);
\item
  ℳ = (𝑀, …) and 𝒩 = (𝑁 , …), 𝑆-structures (aka 𝐿-structures);
\item 
  𝑥 = a tuple of variable symbols;
\item
  φ = φ(𝑥), an 𝐿-formula.
\end{itemize}

\subsection{Expansion by Constants}\label{expansion-by-constants}

\begin{enumerate}
\item
  A \defn{new constant symbol} for 𝐿 is any symbol not occuring in the alphabet of 𝐿.
\item
  If 𝑐 is a set of new constant symbols, then 𝐿(𝑐) is the \defn{expansion of} 𝐿
  \defn{by new constants} and is defined to be the (uniquely determined) language of
  the signature (𝑪 ∪ 𝑐, 𝑭, 𝑹, ρ).
\item
  Δ(𝑐) is the \defn{expansion of} Δ \defn{by new constants} (from 𝑐, where 𝑐 ∩ Δ
  = ∅) and is defined to be the class of formulas obtained from formulas φ ∈ Δ
  upon substituting (at will) elements from 𝑐 for variables in φ. (``At will''
  indicates that Δ ⊆ Δ(𝑐).)
\item
  \(\mathbf{lt}_{L(M)}:=\) the set of all atomic and negated atomic 𝐿(𝑀)-formulas.
\item
  \(\mathbf{cl}_{L(M)}:=\) the set of all atomic and negated atomic 𝐿(𝑀)-setnences.
\end{enumerate}

% \begin{center}\rule{0.5\linewidth}{\linethickness}\end{center}

\subsection{Validity and Truth}\label{validity-and-truth}

\(ℳ \models φ\) means that φ is \defn{valid} in ℳ which means that, for every
tuple 𝑎 from 𝑀 that is at least as long as 𝑥, the 𝐿-sentence φ(𝑎) is \emph{true in} ℳ.

\begin{quote}
\textit{Question}. What is meant by ``φ(𝑎) is true in ℳ?''
\end{quote}

Intuitively, for each 𝑖 we subsitute the element 𝑎 𝑖 for the
variable 𝑥 𝑖 in the formula φ(𝑥), which yields a sentence φ(𝑎) that is
``decidable'' in ℳ. That is, there is a finite procedure by which we can
determine whether or not φ(𝑎) holds (is ``true'') in ℳ.

Formally, however, we may follow a more careful procedure for judging the truth
of a given formula φ in a given structure ℳ. This is the ``simple'' matter of how
syntactically to denote interpretation of variables. As Hodges puts it (in~\cite{Hodges:1993}), this
is ``one of the more irksome parts of model theory.'' The issue is explained
clearly in~\cite[Section 1.4]{Hodges:1993}, so we defer to that treatment:

\begin{quote}
  ``Instead of interpreting a variable symbol as a name of an element 𝑏,
  we can add a new constant for 𝑏 to the signature. The price we pay is that the language
  changes every time another element is named. When constants are added to a
  signature, the new constants and the elements they name are called parameters.\\[6pt]
For example, suppose that 𝒜 is an 𝐿-structure, 𝑎 is a sequence of elements of 𝒜,
and we want to name the elements in 𝑎. Then we choose a sequence 𝑐 of distinct
new constant symbols of the same length as 𝑎, and we form the signature 𝐿(𝑐) by
adding the constants 𝑐 to 𝐿. Then (𝒜 , 𝑎) is an 𝐿(𝑐)-structure and, for each 𝑖,
the element 𝑎 𝑖 is \((𝑐\, i)^{(𝒜, 𝑎)}\), which is the interpretation of 𝑐 𝑖 in
the structure (𝒜 , 𝑎).\\[6pt]
Likewise if ℬ is another 𝐿-structure and 𝑏 a sequence of elements of ℬ of the
same length as 𝑐, then there is an 𝐿(𝑐)-structure (ℬ, 𝑏) in which these same
constants in 𝑐 name the elements of 𝑏. The next lemma is about this situation.
It comes straight out of the definitions and it is often used silently.''
\end{quote}

\begin{lemma}[\protect{\cite[1.4.1]{Hodges:1993}}] Let 𝒜, ℬ be 𝐿-structures and
  suppose (𝒜, 𝑎), (ℬ, 𝑏) are 𝐿(𝑐)-structures. Then a homomorphism 
  ℎ : (𝒜, 𝑎) → (ℬ, 𝑏) is the same thing as a homomorphism ℎ : 𝒜 → ℬ
  such that ℎ ∘ 𝑎 = 𝑏.
\end{lemma}
\begin{remarks*}\
\begin{enumerate}
\item 
  ℎ ∘ 𝑎 denotes the function whose value at 𝑖 is ℎ (𝑎 𝑖);
\item \(φ_x(𝑎)\) denotes [𝑎/𝑥]φ(𝑥) = [𝑎0/𝑥0 , 𝑎1/𝑥1, …]φ(𝑥), which is the sentence
  obtained from φ by substitution: for each 𝑖 replace 𝑥 𝑖 with 𝑎 𝑖.
\item 
  If 𝑡(𝑥) is an 𝐿-term, then \(t^𝒜 (𝑎)\) and \(t(𝑐)^{(𝒜, 𝑎)}\) are the same
  element.
\item
  If φ(𝑥) is an atomic formula then \(𝒜 \models φ_x(𝑎) \; ⇔ \; (𝒜, 𝑐) \models φ_x(𝑐)\).
\end{enumerate}
\end{remarks*}

\section{Models, Theories, Diagrams}\label{models-theories-diagrams}

Let φ ∈ 𝐿₀, ℰ ⊆ 𝐿₀, and let ℳ = (𝑀, …), 𝒩 = (𝑁, …) be 𝐿-structures. Let Δ be an
arbitrary class of formulas (not necessarily from 𝐿).

\subsection{Models}\label{models}

\begin{enumerate}
\item
  If 𝑀 ≠ ∅ and \(ℳ \models ℰ\), then ℳ is a \defn{model} of ℰ; we also say
  ``ℳ \emph{models} ℰ.''
\item
  \(\operatorname{Mod}_L ℰ :=\) the class of 𝐿-structures that model ℰ.
\item
  \(\operatorname{Mod}_L ∅ :=\) the class of all nonempty 𝐿-structures.
\item
  ℰ \defn{entails} φ, denoted ℰ ⊢ φ, just in case every model of ℰ also models φ.
\item
  φ is a \defn{logical consequence} of ℰ just in case ℰ entails φ.
\item
  The \defn{deductive closure} of ℰ is the set \(ℰ^⊢ = \{φ ∈ 𝐿₀ : ℰ ⊢ φ\}\) of
  logical consequences of ℰ.
\item
  ℰ is \defn{deductively closed} provided \(ℰ^⊢ ⊆ ℰ\).
\item
  The sets ℰ₀, ℰ₁ ⊆ 𝐿₀ are ℰ-\defn{equivalent} if \((ℰ ∪ ℰ₀)^⊢ = (ℰ ∪ ℰ₁)^⊢\).
\item
  \defn{logically equivalent} means ∅-equivalent.
\item
  A \defn{contradiction} is an 𝐿-sentence of the form φ ∧ ¬ φ.
\item
  ℰ is \defn{consistent} if \(ℰ^⊢\) contains no contradictions; otherwise, ℰ is
  \defn{inconsistent}.
\end{enumerate}
\begin{remark*}
No model satisfies a contradiction, so the deductive closure of
a contradiction is the set 𝐿₀ of all 𝐿-sentences, and 𝐿₀ is the only deductively
closed inconsistent set of 𝐿-sentences.
\end{remark*}

\subsection{Theories}\label{theories}

\begin{enumerate}
\item
  An 𝐿-\defn{theory} is a \emph{consistent} and \emph{deductively closed} set of
  𝐿-sentences.
\item
  The \defn{cardinality} or \defn{power} of an 𝐿-theory 𝑇 (denoted ∣𝑇∣) is the cardinality of 𝐿.
\item
  \(T_Δ = (T ∩ Δ)^⊢\), the Δ-\defn{part} of the 𝐿-theory 𝑇 (here Δ is an
  arbibtrary class of formulas).
\item
  \(\bfall\) is the class of formulas in which \(\exists\) does not appear;
  \(T_{\bfall} = (T ∩ \bfall)^⊢\),  the \emph{universal part} of 𝑇.
\item
  \(\operatorname{Th}_Δ ℳ := \{φ ∈ 𝐿₀ : φ ∈ Δ, \, ℳ ⊢ φ\}\),
  the set of 𝐿-sentences in Δ that are true in ℳ.
\item
  \(\operatorname{Th} ℳ := \operatorname{Th}_{L_0} ℳ\),
  the set of 𝐿-sentences true in ℳ.
\item
  An 𝐿-theory 𝑇 is \defn{complete} if for all φ ∈ 𝐿₀, either φ ∈ 𝑇 or ¬ φ ∈ 𝑇.
\end{enumerate}

\begin{lemma}[\protect{\cite[3.5.1]{Rothmaler:2000}}] If 𝑇 is an 𝐿-theory, the
  following are equivalent:
  \begin{enumerate}
  \def\labelenumi{\arabic{enumi}.}
  \item 𝑇 is complete.
  \item 𝑇 is a maximal 𝐿-theory.
  \item 𝑇 is a maximal consistent set of 𝐿-sentences.
  \item 𝑇 = Th ℳ for all ℳ ⊢ 𝑇.
  \item 𝑇 = Th ℳ for some ℳ ⊢ 𝑇.
  \end{enumerate}
\end{lemma}

\begin{examples*}
\(T^∞\) is the theory of the class of all infinite models of 𝑇, and 𝑇₌ is the \defn{theory of pure identity}, which is the 𝐿₌-theory of all sets (regarded as 𝐿₌-structures). 
\end{examples*}
\subsection{Diagrams}\label{diagrams}
\begin{enumerate}
\item The \defn{diagram} of ℳ is the set \(𝐷(ℳ) := \mathbf{cl}_{𝐿(𝑀)}\) of all
  atomic and negated atomic 𝐿(𝑀)-sentences;
\item
  \(ℳ ⇒_Δ 𝒩\) means ∀ φ ∈ Δ ∩ 𝐿₀ (ℳ ⊢ φ → 𝒩 ⊢ φ).
\item
  ℳ ⇒ 𝒩 means \(ℳ ⇒_𝐿 𝒩\).
\item
  ℳ ≡ 𝒩 means ℳ ⇒ 𝒩 and ℳ ⇐ 𝒩 hold, and this is equivalent to Th ℳ = Th 𝒩.
  We call ℳ and 𝒩 \defn{elementarily equivalent} in this case.
\item
  𝑓 : ℳ ↪ 𝒩 means 𝑓 is an 𝐿-structure-monomorphism from ℳ to 𝒩.
\item
  \(𝑓 : ℳ \stackrel{Δ}{⟶} 𝒩\) means all 𝐿-formulas in Δ are preserved by 𝑓; that
  is, ℳ~⊢~φ(𝑎) implies 𝒩~⊢~φ(𝑓~∘~𝑎), for all φ ∈ Δ ∩ 𝐿 and all tuples 𝑎 from 𝑀.
\item
  \(𝑓 : ℳ \stackrel{≡}{⟶} 𝒩\) means \(𝑓 : ℳ \stackrel{L}{⟶} 𝒩\).
\end{enumerate}

\subsection{Some Facts}\label{some-facts}
Let ℳ and 𝒩 be 𝐿-structures and let Δ be a set of 𝐿-formulas.  If 𝑓 : 𝑀 → 𝑁,
then denote by 𝑓[𝑀] the \defn{image} of 𝑀 under 𝑓.
\begin{enumerate}
\item 
  \(f : ℳ \stackrel{\mathbf{at}}{⟶} 𝒩\) iff 𝑓 : ℳ → 𝒩
\item 
 \(f : ℳ \stackrel{Δ}{⟶} 𝒩\) iff \((ℳ, M) \Rrightarrow_{Δ(M)} (𝒩, f[M])\).
\item If \(f : ℳ \stackrel{Δ}{⟶} 𝒩\) and Δ contains \(\mathbf{at}\) and all
  negations of unnested relational atomic formulas, then 𝑓 is a strong
  homomorphism. (The converse is not true.)
\item If \(f : ℳ \stackrel{Δ}{⟶} 𝒩\) and Δ is closed under negation, then
  ℳ ⊢ φ(𝑎) implies 𝒩 ⊢ φ(𝑓 ∘ 𝑎), for all φ~∈~Δ and tuples 𝑎 from 𝑀.
\item 𝑓 : 𝑀 → 𝑁 is injective iff \(f : ℳ \stackrel{Δ}{⟶} 𝒩\)
for the set \(Δ = \{x \neq y\}\).
\item If Δ ⊆ 𝐿₀, then \(f : ℳ \stackrel{Δ}{⟶} 𝒩\) iff \(ℳ ⇒_Δ 𝒩\) and
𝑓 : 𝑀 → 𝑁.
\item If Δ ⊆ 𝐿₀ and Δ is closed under negation, then \(ℳ ⇒_Δ 𝒩\) implies \(ℳ ≡_Δ 𝒩\).
\end{enumerate}


\subsection{The Lemma on Constants}\label{the-lemma-on-constants}

Above we remarked that if (𝒜, 𝑎) is an 𝐿(𝑐)-structure with 𝒜 an 𝐿-structure,
then for every atomic formula φ of 𝐿, we have 𝒜 ⊢ φ(𝑎) if and only if (𝒜, 𝑎) ⊢ φ(𝑐).

\begin{lemma}[\protect{\cite[2.3.2]{Hodges:1993}}] Let 𝐿 be a language, 𝑇 a
theory in 𝐿, and φ(𝑥) a formula in 𝐿. Let 𝑐 be a sequence of distinct constants
that are not in 𝐿. Then 𝑇 ⊢ φ(𝑐) if and only if 𝑇 ⊢ ∀ 𝑥, φ.
\end{lemma}

\subsection{The Diagram Lemma}\label{the-diagram-lemma}

\begin{lemma}[\protect{\cite[6.1.2]{Rothmaler:2000}}]
  Let ℳ and 𝒩 be 𝐿-structures.
  \begin{enumerate}
  \item
    \(𝑓 : ℳ ↪ 𝒩 \, ⇔ \, 𝑓 : ℳ \stackrel{\mathbf{qf}}{⟶} 𝒩
                  \, ⇔ \, (𝒩, 𝑓[M]) ⊢ 𝐷(ℳ)\).
    In particular \(f : ℳ \stackrel{≡}{↪} 𝒩  \, ⇒  \, f : ℳ ↪ 𝒩\).
  \item
    \(ℳ ↪ 𝒩 \; ⇔ \; 𝒩\) has an 𝐿(𝑀)-expansion that models 𝐷(ℳ).
  \end{enumerate}
\end{lemma}

\subsection{The Diagram Lemma (ver. 2)}\label{the-diagram-lemma-ver.-2}
Let's consider an alternative version of the Diagram Lemma that makes the role
of new constants more explicit. For this version, we will introduce some 
new notation.

\begin{itemize}
\item 𝑐 is a tuple of distinct symbols not appearing in 𝐿;
\item \(𝑎 : \{0, 1, …, 𝑛-1\} → 𝑀\) and \(𝑏 : \{0, 1, …, 𝑛-1\} → 𝑁\) are 𝑛-tuples in 𝑀 and 𝑁 respectively.
\item (ℳ, 𝑐) and (𝒩, 𝑐) are 𝐿(𝑐)-structures, where \((𝑐 \,i)^ℳ = 𝑎 \, i\) and
  \((𝑐\, i)^𝒩 = 𝑏\, i\) are the interpretations in ℳ and 𝒩 of the new constant
  symbols;
\item
  ⟨ 𝑎 ⟩ is the substructure of ℳ generated by the elements of the tuple 𝑎;
\end{itemize}

\begin{lemma}[\protect{\cite[1.4.2]{Hodges:1993}}] The following are equivalent:
  \begin{enumerate}
  \item For every atomic sentence φ(𝑐) of 𝐿(𝑐), if (ℳ, 𝑐) ⊢ φ(𝑐) then (𝒩, 𝑐) ⊢ φ(𝑐). 
  \item There is a homomorphism 𝑓 : ⟨ 𝑎 ⟩ → 𝒩 such that 𝑓 ∘ 𝑎 = 𝑏.
  \item The homomorphism is unique, if it exists, and it is an embedding if and
    only if for every atomic sentence φ of 𝐿(𝑐), we have (ℳ, 𝑐) ⊢ φ  ⇔ (𝒩, 𝑐) ⊢ φ.
  \end{enumerate}
\end{lemma}

\section{Elementary equivalence}\label{elementary-equivalence}
\subsection{Isomorphic structures are elementarily equivalent}\label{isomorphic-structures-are-elementarily-equivalent}
\begin{lemma}[\protect{\cite[6.1.3]{Rothmaler:2000}}]
If 𝑓 : ℳ ≅ 𝒩, then \(𝑓 : ℳ \stackrel{≡}{↪} 𝒩\), hence also ℳ ≡ 𝒩.
\end{lemma}
The second implication is \(𝑓 : ℳ \stackrel{≡}{↪} 𝒩\) implies ℳ ≡ 𝒩. 
That's true because, if \(𝑓 : ℳ \stackrel{≡}{↪} 𝒩\), then we have not only
φ ∈ Th ℳ implies φ ∈ Th 𝒩, but also ¬ φ ∈ Th ℳ implies ¬ φ ∈ Th 𝒩. From the
former, ℳ ⇒ 𝒩. From the latter, ℳ ⇐ 𝒩. Thus, ℳ ≡ 𝒩.
(For proof of the first implication, see \cite[page 70]{Rothmaler:2000}.)
The converse of 6.1.3 holds if and only if the structures involved are
finite, as the next proposition shows.

\begin{lemma}[\protect{\cite[8.1.1]{Rothmaler:2000}}]
Let ℳ be an 𝐿-structure. The following are equivalent:
\begin{enumerate}
\item For every 𝐿-structure 𝒩, 𝒩 ≡ ℳ implies 𝒩 ≅ ℳ.
\item ℳ is finite.
\end{enumerate}
\end{lemma}

\subsection{All models of a complete theory are elementarily equivalent}\label{all-models-of-a-complete-theory-are-elementarily-equivalent}

\begin{lemma}[\protect{\cite[8.1.2]{Rothmaler:2000}}]
 A theory is complete iff its models are elementarily equivalent.
\end{lemma}

\begin{corollary}[\protect{\cite[8.1.3]{Rothmaler:2000}}]
 A complete theory has a finite model iff it has only one model up to isomorphism.
\end{corollary}

\section{Elementary maps}\label{elementary-maps}
Let ℳ and 𝒩 be 𝐿-structures. A map 𝑓 : 𝑀 → 𝑁 is called \defn{elementary} if
\(f : ℳ \stackrel{≡}{↪} 𝒩\).
The structure ℳ is \defn{elementarily embeddable} in 𝒩, in symbols
\(ℳ \stackrel{≡}{↪} 𝒩\), if there is an elementary map \(f : ℳ \stackrel{≡}{↪} 𝒩\).\\[6pt]
\begin{remarks*}\
\begin{enumerate}
\item
  If \(f : ℳ \stackrel{≡}{↪} 𝒩\), then ℳ ≡ 𝒩.
\item
  While elementary equivalence is weaker than isomorphism, every elementary
  map is automatically an isomorphic embedding
  (by~\cite[Lemma 6.1.2(1)]{Rothmaler:2000}). Therefore, elementary maps are
  aka \defn{elementary embeddings}, and the notation
  \(f : ℳ \stackrel{≡}{↪} 𝒩\) is justified.
\item
  The converse is not true in general, unless the embedding is surjective
  (i.e., an isomorphism), since every isomorphism \(f : ℳ \cong 𝒩\) is an
  elementary map (by~\cite[Prop 6.1.3]{Rothmaler:2000}).
\end{enumerate}
\end{remarks*}

\subsection{Elementary Diagram Lemma}\label{elementary-diagram-lemma}

The \defn{elementary diagram} of an 𝐿-structure ℳ is the complete 𝐿(𝑀)-theory Th(ℳ, 𝑀).

\begin{lemma}[\protect{\cite[8.2.1]{Rothmaler:2000}}]
 Let ℳ and 𝒩 be 𝐿-structures and 𝑓 : 𝑀 → 𝑁.
\begin{enumerate}
\item  𝑓 is elementary ⇔ (ℳ, 𝑀) ≡ (𝒩, 𝑓[𝑀]) ⇔ (𝒩, 𝑓[𝑀]) ⊢ Th(ℳ, 𝑀).
\item \(ℳ \stackrel{≡}{↪} 𝒩\) ⇔ 𝒩 has an expansion that is a model of Th(ℳ, 𝑀).
\end{enumerate}
\end{lemma}
%%%%%%%%%%%%%%%%%%%%%%%%%%%%%%%%%%%%%%%%%%%%%%%%%%%%%%%%%%%%%%%%%%%%%%%%%%%%%%%%%%%%%%%%%% 
%%%%%%%%%%%%%%%%%%%%%%%%%%%%%%%%%%%%%%%%%%%%%%%%%%%%%%%%%%%%%%%%%%%%%%%%%%%%%%%%%%%%%%%%%%

%% Bibliography
\bibliographystyle{plainurl}% the mandatory bibstyle

\bibliography{mtrefs}


\end{document}